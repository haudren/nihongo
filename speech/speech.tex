\documentclass[a4paper,12pt]{article}

\usepackage{fullpage}
\usepackage{CJK}
\usepackage{setspace}
\usepackage{ruby}
\doublespacing
\renewcommand{\rubysep}{-1ex}
\begin{document}
\begin{CJK}{UTF8}{min}

%\linespread{2}
\author{Herv\'e Audren - オドレン エルヴェ}
\title{日本語のスピーチ}
\date{平成25年 9月 13日}
\maketitle
\paragraph*{}\ruby{皆}{みんな}さん、\ruby{今晩}{こんばん}は。
\paragraph*{}\ruby{明後日}{あさって}フランスへ\ruby{帰}{かえ}るしまいました。\ruby{日本}{にほん}に\ruby{六ヶ月}{ろっかげつ}\ruby{前}{まえ}\ruby{来}{き}た、みじかい\ruby{時間}{じかん}でしたが \ruby{新}{あたら}しい\ruby{友達}{ともだち}に\ruby{会}{あ}うし、\ruby{日本語}{にほんご}を\ruby{勉強}{べんきょう}するし、ロボットを\ruby{研究}{けんきゅう}するし、すばらしいです。フランスへ\ruby{来}{く}れば、さびしいになりますが、\ruby{一月}{いちがつ}に\ruby{日本}{にほん}に\ruby{来}{く}る\ruby{予定}{よてい}でうれしいです。

\paragraph*{}\ruby{日本}{にほん}の\ruby{文化}{ぶんか}とフランスは\ruby{全然}{ぜんぜん}\ruby{違}{ちが}う。たとえば、\ruby{子供}{こども}の\ruby{時}{とき}、かぞくと\ruby{食事}{しょくじ}をしながら、\ruby{騒}{さわ}げば、\ruby{父}{ちち}は僕にしかれました。ここ、パンは\ruby{軟}{やわ}らかいですがフランス人は\ruby{硬}{かた}くて\ruby{好}{す}きです。
\ruby{料理}{りょうり}について、フランスの\ruby{料理}{りょうり}は、\ruby{日本}{にほん}の\ruby{料理}{りょうり}よりもっと人気があるでも、レストランが\ruby{安}{やす}いし、\ruby{生}{なま}\ruby{魚}{さかな}を\ruby{食}{た}べられるし、いつも日本の\ruby{食}{た}べ\ruby{物}{もの}\ruby{食}{た}べたいです。でも、\ruby{全部}{ぜんぶ}の\ruby{食}{た}べ\ruby{物}{もの}\ruby{好}{す}きじゃない。たとえば、なっとうがきらくて、\ruby{温泉}{おんせん}にはだかではずかしいです。\ruby{結論}{けつろん}を\ruby{言}{い}えば、まだ\ruby{日本}{にほん}\ruby{人}{じん}になりませんがたぶん\ruby{来年}{らいねん}\ruby{全部}{ぜんぶ}の\ruby{料理}{りょうり}\ruby{食}{た}べられるようになるでしょう!

\paragraph*{}ありがとうございました!

\end{CJK}
\end{document}