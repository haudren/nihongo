\documentclass[a4paper,12pt]{article}

\usepackage{fullpage}
\usepackage{CJK}
\usepackage{setspace}
\usepackage{ruby}
\doublespacing
\renewcommand{\rubysep}{-1ex}
\begin{document}
\begin{CJK}{UTF8}{min}

%\linespread{2}
\author{Herv\'e Audren - オドレン エルヴェ}
\title{日本語のスピーチ}
\date{平成25年 9月 13日}
\maketitle
\paragraph*{}\ruby{皆}{みんな}さん、\ruby{今晩}{こんばん}は。
\paragraph*{}\ruby{私}{わたし}はもう、\ruby{明後日}{あさって}フランスへ\ruby{帰}{かえ}ってしまいます。\ruby{日本}{にほん}には\ruby{六ヶ月}{ろっかげつ}\ruby{前}{まえ}に\ruby{来}{き}ました、みじかい\ruby{時間}{じかん}でしたが \ruby{新}{あたら}しい\ruby{友達}{ともだち}に\ruby{会}{あ}えてし、\ruby{日本語}{にほんご}も\ruby{勉強}{べんきょう}できるし、ロボットも\ruby{研究}{けんきゅう}できるし、すばらしかったです。フランスへ\ruby{帰}{かえ}ると、さびしくになりますが、\ruby{一月}{いちがつ}にまた\ruby{日本}{にほん}に\ruby{来}{く}る\ruby{予定}{よてい}なのでうれしいです。

\paragraph*{}\ruby{日本}{にほん}とフランスの\ruby{文化}{ぶんか}は\ruby{全然}{ぜんぜん}\ruby{違}{ちが}います。たとえば、\ruby{子供}{こども}の\ruby{時}{とき}、かぞくと\ruby{食事}{しょくじ}をしながら、\ruby{騒}{さわ}げば、\ruby{父}{ちち}はしかられましたが\ruby{日本}{にほん}ではだいじょうぶです。さらに、\ruby{日本}{にほん}のパンは\ruby{軟}{やわ}らかいですがフランス人は\ruby{硬}{かた}い\ruby{方}{ほう}が\ruby{好}{す}きです。
\ruby{料理}{りょうり}について\ruby{言}{い}うと、フランスの\ruby{料理}{りょうり}は、\ruby{日本}{にほん}の\ruby{料理}{りょうり}よりもっと人気がありますが、\ruby{日本}{にほん}はレストランが\ruby{安}{やす}いし、\ruby{生}{なま}\ruby{魚}{さかな}が\ruby{食}{た}べられるし、いつも日本の\ruby{食}{た}べ\ruby{物}{もの}を\ruby{食}{た}べたいと\ruby{思}{おも}います。でも、\ruby{全部}{ぜんぶ}の\ruby{食}{た}べ\ruby{物}{もの}は\ruby{好}{す}きじゃない。たとえば、なっとうがきらいです。そして、\ruby{温泉}{おんせん}にはだかで\ruby{入}{はい}るのははずかしいです。\ruby{結論}{けつろん}を\ruby{言}{い}えば、まだ\ruby{日本}{にほん}\ruby{人}{じん}になりませんがたぶん\ruby{来年}{らいねん}には\ruby{全部}{ぜんぶ}の\ruby{料理}{りょうり}が\ruby{食}{た}べられるようになるでしょう!

\paragraph*{}ありがとうございました!

\end{CJK}
\end{document}