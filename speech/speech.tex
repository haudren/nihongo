\documentclass[a4paper,11pt]{article}

\usepackage{fullpage}
\usepackage{CJK}
\usepackage{setspace}


\begin{document}
\begin{CJK}{UTF8}{min}
\doublespacing
\author{Herv\'e Audren - オドレン エルヴェ}
\title{日本語のスピーチ}
\date{平成25年 9月 13日}
\maketitle

\paragraph*{}皆さん、今晩は。
\paragraph*{}明後日フランスへ帰るしまいました。日本に六ヶ月前来た、みじかい時間でしたが 新しい友達に会うし、日本語を勉強するし、すばらしいです。フランスへ来れば、さびしいになりますが、一月に日本に来る予定でうれしいです。

\paragraph*{}日本の文化とフランスは全然違う。たとえば、子供の時、かぞくと食事をしながら、騒げば、父は僕にしかれました。ここ、パンは軟らかいですがフランス人は硬くて好きです。

\paragraph*{}子供の時、フランスの東南方にすんでいました。グルノブルと言う町です。

\end{CJK}
\end{document}