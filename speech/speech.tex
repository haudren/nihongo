\documentclass[a4paper,11pt]{article}

\usepackage{fullpage}
\usepackage{CJK}
\usepackage{setspace}


\begin{document}
\begin{CJK}{UTF8}{min}
\doublespacing
\author{Herv\'e Audren - オドレン エルヴェ}
\title{日本語のスピーチ}
\date{平成25年 9月 13日}
\maketitle

\paragraph*{}皆さん、今晩は。
\paragraph*{}明後日フランスへ帰るしまいました。日本に六ヶ月前来た、みじかい時間でしたが 新しい友達に会うし、日本語を勉強するし、ロボットを研究するし、すばらしいです。フランスへ来れば、さびしいになりますが、一月に日本に来る予定でうれしいです。

\paragraph*{}日本の文化とフランスは全然違う。たとえば、子供の時、かぞくと食事をしながら、騒げば、父は僕にしかれました。ここ、パンは軟らかいですがフランス人は硬くて好きです。料理について、フランスの料理は、日本の料理よりもっと人気があるでも、レストロランが安いし、生魚を食べられるし、いつも日本の食べ物食べたいです。でも、全文の食べ物好きじゃない。たとえば、なっとうがきらくて、温泉にはだかではずかしいです。結論を言えば、まだ日本人になりませんがたぶん来年全部の料理食べられるようになるでしょう!

\paragraph*{}ありがとうございました!

\end{CJK}
\end{document}